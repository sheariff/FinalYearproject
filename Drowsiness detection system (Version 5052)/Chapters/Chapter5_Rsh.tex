\chapter{Conclusions and Future Work}
\label{chap:conclusions}
\lhead{\emph{Conclusions}}
\section{Overview}
We now have some understanding of what a Drowsiness detection system is and this is very good,  as I enter into the implementation phase of my project things that I was confused about before I started my research have been clarified, certain areas I felt that I had a vague understanding of such things such as, how I will detect a persons face in a image. These self question marks have  have been clarified. Now I know that machine learning algorithms must be used and trained in order to achieve this task. 

\section{Discussion}
A major problem that I can reflect on during this Phase of the project was my time management, i honestly have to say that this must change and i must have a plan in place for the implementation phase of the project so that I can successfully achieve the requirements of my system without being stuck for time. Projects from other modules have played a major role in the delay of this being completed in a timely manner. If I used my time more wisely I feel that this project would have been completed with a greater quality and with much greater detail, however I must say I do feel accomplished as this is a very important piece of work and major work has gone into completing this research phase.

\section{Conclusion}
During the Background research I felt that this was the most essential piece of work that had to do as it clarified everything about the project that has been done in the past and present, also providing me with a very important understanding of how this project must be carried out. The background research portion helped me me learn large amounts of information about drowsiness detection systems and expand my knowledge on software solutions present that can aid with the implementation of my project.
During the problem definition portion of my project the main thing that I have taken away from completing this chapter is that I was able to pin point the exact problem that my project will be targeting when I'm implementing it in the future, This chapter helped my identify the best functional and non functional characteristics that my system must present in the final design. This chapter was the chapter that helped me identify new goals for my drowsiness detection system.
During my work on the  implementation approach I achieved many things, firstly i felt that this was the toughest chapter to complete as there were many questions I had to ask myself and a lot of decisions I had to make but the result of this was a very detailed mental visualisation of how I will develop my Project in the Implementation phase in semester 2, This chapter gave me different perspective on how I visualised my Project from how I will carry out my project to how I will mitigate the risks forcing me to develop a prototypes to that I can feel assured that the Technologies I have chosen are functional in my computer and thankfully they are.


\section{Future Work}

Overall I did achieve many things during this research phase but here have been many other things that I haven't achieved and wish I was able to accomplish. If I had more time allocated i would have added far more images and diagrams so that I can help explain and summarise a-lot of the explanations that I have provided through out this research phase, if i had more time I would have liked to added more to the Bibliography  in order to have a much fuller Background research and expand my knowledge even further on the Research undergone with Drowsiness detection, maybe if I had a bit more time i would have added more technologies and explored other technologies that i would have found interesting. Overall if i had more time i would have tried to build a prototype of a machine learning algorithm.





