\chapter{Introduction}
\label{chap:intro}
\lhead{\emph{Introduction}}\textbf{}

For my final year project I have tasked myself with developing a drowsiness detection system, with the objective of being able to track a persons eyes or facial expressions for cues that would lead to the detection of drowsiness. The Drowsiness detection system will use eye tracking technology which may be open sourced or developed using certain programming  library's to successfully identify when a person is starting to fall asleep or the persons eyes are starting to close, once the system has detected that the persons eyes have been closed it then emit a warning single to the driver by way of a loud sound that will wake the driver up prevent him from falling asleep.  


\section{Motivation}
Drowsiness occurs when a individual person starts feeling sleepy, its a feeling of sleepiness and is one of leading causes behind vehicle crashes by motorist all over the world. The Road Safety Authority is the leading body responsible keeping the roads safe in Ireland, they try to make roads safer by reducing the number of injuries and the severity of these injuries. 

The road safety authority have stated that the " Drivers suffering from a sleep debt are at risk of ‘nodding off’ whilst driving and substantially increasing their risk of being involved in a crash. It is estimated that driver fatigue is a contributory factor in as many as 1 in 5 driver deaths in Ireland every year. Furthermore, tiredness-related collisions are 3 times more likely to be fatal or result in a serious injury because of the high impact speed and lack of avoiding action. A survey* of drivers’ attitudes to driver fatigue conducted by the Road Safety Authority in 2014 revealed that over 1 in 10 motorists have fallen asleep at the wheel.  The survey also found that motorists who drive as part of their work, and motorists who admit to driving after taking any amount of alcohol, had a higher than average incidence of falling asleep at the wheel (almost 1 in 5 fell asleep at the wheel)". 
Also ( "A study by the AAA Foundation for Traffic Safety estimated that 328,000 drowsy driving crashes occur annually. That's more than three times the police-reported number. The same study found that 109,000 of those drowsy driving crashes resulted in an injury and about 6,400 were fatal"). 

When looking at these local and global figures I always wondered if there was a way to reduce these numbers with the hope of saving lives, lower the amount of crashes and reduce damages to the driver, environment and to the insurance companies that have to pay out money for damages that may have happened to another persons vehicle or property. I do feel these accidents can be avoided with some ingenuity, this has lead me to research and to develop a drowsiness detection system in order to detect drowsiness behind the wheel, also I do wish to develop the drowsiness detection system as a solution for other environments, for example jobs which require people to work at higher altitudes or under higher G-Force such as jobs that require jet flight. These types of jobs are dangerous due to the lower oxygen levels that go along with the tasks that are associated while performing these jobs and due to the decreased concentration levels of oxygen in the air this can result in the person losing consciousness because the human body needs a certain level oxygen within per to be available in there blood and more specifically a good level of oxygen must be received by there persons brain or else the brain will not be able to perform effectively, so I do believe that my motivation with this project is that I can tackle a common local and global problem and also a problem that occurs to humans naturally due to environment that person is in.
\section{Contribution}
During my time in collage I have  taken many modules that aided me towards  the completion of this project such as Linear data structures and algorithms, this module helped me develop better technical programming skills such as being able to use recursion to help aid me when i am developing my projects, this didn't just improve my coding skills it helped me develop a stronger coding sense. object orientated principles/ object orientated programming module,  both of these modules have taught me the fundamentals of programming in java which have helped me to easily understand other programming languages such as C and Python very easily as the logic used in these languages is very similar to java not only by learning these language, undertaking both of these modules have allowed me understand things such as inheritance, serialization, super classes and the ability to develop GUI's and techniques to help send information to a back end database such as MY-SQL. Programming Data analytic was another excellent module that is crucial to the completion of my Drowsiness detection system as python was taught in this module and this is the core programming language used to develop my project. python is Versatile, Easy to Use and Fast to Develop with, contains many libraries and technologies that helped me develop my Drowsiness detection system. Distributed system programming aided me by teaching me how to communicate between two different application (Server and client)  to fulfill tasks by sending information between them along with multi-threading/synchronization by the means of messages. 

\section{Structure of This Document}
When looking through this document You will notice that most of the work done in this Document will be the chapters 2 and 4 while the remaining chapter 1  chapter 3 and chapters 5 consists of work done by myself, but does not have as much work as the other chapters.
\subsection{Chapter 2}
In chapter 2  The chapter starts off with how the drowsiness detection system places itself within the thematic area in computer science this section consists of how This system positions itself within computer science. The chapter then goes on to detail all the research paper, blogs and websites that have been read and cited followed by the state of the art which is what is being done with the drowsiness detection system today actively.
\subsection{Chapter 3}
This chapter is far more brief and the issue tackled in this chapter is the problem that the drowsiness detection system will tackles the requirements that it must achieve, the functional and non functional requirements and the objective that the systems  must achieve.
\subsection{Chapter 4}
This chapter is basically a detail of how the drowsiness detection system will be designed and implemented going forward with the implementation of the system. The chapter is broken up into  headings such as architecture which has subheadings consisting of design models, technologies, databases and frameworks of the drowsiness detection system. followed by another heading called risks. Within risks there is a detail of all the risks that this system could encounter while going forward with the implementation. Other headings called methodologies and implementation plan schedule will follow shortly as separate heading. In the methodology heading their is discussion on what agile philosophical methods will be used to delegate work in semester 2, while the implementation plan schedule will consist of tasks that I have outlined and will achieve for semester 2, after this there is another 2 separate headings evaluation and prototype. For the evaluation portion you will just find a short summary of how I will analyse the projects success and how i will accept things in this System to be a success during the implementation phase. The prototype heading will consist of a short test of the technologies that my system will be using during the second semester.
\subsection{Chapter 5}
This is a very short chapter and consist mostly of a reflective opinion of the work done on this document as-well as things that i have achieved and didn't. This chapter will consist of a overview,discussion, conclusion and future work. All of these are headings with brief reflections on my journey on work on this project. 




